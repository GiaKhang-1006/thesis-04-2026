\chapter{Thực nghiệm}
\label{Chapter 4}

\section {Giới thiệu thực nghiệm}
% Giới thiệu ngắn gọn mục đích thực nghiệm, nhấn mạnh so sánh EdgeFace (Lightweight) với baseline OPQN, tập trung vào tính tinh gọn và khả năng mở rộng. 

% Cần có:
% Mục tiêu: Đánh giá accuracy, MAP, P@10, speed (ms/query), params.
% Các biến thể: Freeze, unfreeze, CosFace, ArchFace
% Dataset: Facescrub seen (chi tiết số ảnh/danh tính), unseen/VGGface2
% So sánh: Với OPQN, nhấn mạnh đóng góp mới

%K Kế hoạch viết: Bắt đầu bằng đoạn tổng quát, thêm lý do chọn dataset (intra-class variances cao).

Trong chương này, chúng tôi trình bày các thực nghiệm nhằm đánh giá hiệu quả của mô hình EdgeFace - một kiến trúc tinh gọn được đề xuất cho nhiệm vụ truy xuất ảnh mặt người trên tập dữ liệu lớn - so với mô hình cơ sở OPQN sử dụng mạng xương sống ResNet20 \cite{opqn}. Các thực nghiệm tập trung vào hai khía cạnh chính: (i) độ chính xác truy xuất, được đo bằng chỉ số trung bình của độ chính xác trung bình (MAP) và Độ chính xác trong T kết quả hàng đầu (P@T), nhằm đánh giá khả năng trả về các kết quả liên quan một cách chính xác; và (ii) tốc độ truy vấn, được đo lường bằng thời gian trung bình mỗi truy vấn(ms/query), nhằm nhấn mạnh tính hiệu quả thực tiễn của mô hình trong môi trường dữ liệu lớn với tài nguyên hạn chế. 

Để đạt được mục tiêu trên, chúng tôi thử nghiệm các biến thể của EdgeFace, bao gồn: Freeze Backbone (giữ cố định các lớp thấp để giảm quá khớp và tăng tốc độ huấn luyện), Unfreeze Backbone (huấn luyện toàn bộ mạng xương sống để học các đặc trưng sâu hơn), huấn luyện lại với CoseFace (xử dụng biên độ cosine để tăng khả năng phân biệt), và huấn luyện lại với ArcFace (sử dụng biên độ góc để cải thiện khả năng tổng quát hoá trên các danh tính chưa từng thấy). Các thực nghiệm hiện tại chủ yếu được thực hiện trên tập dữ liệu FaceScrub với thiết lập các danh tính đã tâhys, nơi tập huấn luyện và kiểm tra chia sẻ cùng các danh tính, nhằm đánh giá hiệu suất cơ bản của mô hình. Các kết quả trên các danh tính chưa thấy (kiểm tra trên các danh tính không có trong huấn luyện) và tập dữ liệu VGGFace2 đang được cập nhật và sẽ được bổ sung theo hướng nghiên cứu trong bài báo gốc \cite{opqn}.

Phần thực nghiệm nhằm chứng minh rằng EdgeFace không chỉ tinh gọn hơn về số lượng tham số (Giảm khoảng 92\% so với ResNet20) mà còn có khả năng duy trì hoặc cải thiện hiệu xuất so với mô hình cơ sở, đặc biệt trong bối cảnh truy xuất ảnh mặt người với các biến thể ngoại lớp lớn (kiểu dáng, ánh sáng, biểu cảm) và các khoảng cách nội lớp nhỏ. Chúng tôi sẽ phân tích chi tiết từng khía cạnh, bao gồm lý do cho các kết quả bất ngờ (như MAP thấp ở một số biến thể), các đánh đổi giữa độ chính xác và tốc độ, cũng như đề xuất các hướng tối ưu hoá để khắc phục hạn chế. Các kết quả sẽ được trình bày qua bảng biểu và hình ảnh minh hoạ, với phân tích liên hệ chặt chẽ đến mục tiêu đề tài: Phát triển một kiến trúc tinh gọn phù hợp cho truy xuất ảnh mặt người trên tập dữ liệu có khả năng mở rộng.

\section {Thiết lập thực nghiệm}

\section {Kết quả và phân tích}

\section {Nghiên cứu cắt bỏ}