\chapter*{Tóm tắt}
\label{summary}

Giới hạn 200-300 từ, sử dụng văn phong học thuật, ngắn gọn, khách quan, viết liền mạch, không gạch đầu dòng.

Hướng dẫn trình bày tóm tắt:

\begin{enumerate}
    \item Giới thiệu vấn đề/ngữ cảnh của vấn đề (khoảng 1 - 3 câu): Trình bày bối cảnh và vấn đề thực tế đang tồn tại/ khoảng trống nghiên cứu, tầm quan trọng của vấn đề nghiên cứu, lý do cần thực hiện đề tài.
    \item Mục tiêu đề tài (1 câu): nêu cụ thể đề tài giải quyết/phát triển điều gì.
    \item Phương pháp và giải pháp (1 - 3 câu): Trình bày phương pháp, công nghệ, công cụ, mô hình, giải pháp được đề xuất trong đề tài. Nêu những cải tiến nổi bật (nếu có). Nêu phương pháp thử nghiệm, đánh giá, bộ dữ liệu được sử dụng để đánh giá (nếu có)
    \item Kết quả chính (1 - 2 câu): Trình bày ngắn gọn các kết quả nổi bật, có thể đưa số liệu cụ thể (độ chính xác, tốc độ xử lý, dung lượng dữ liệu, \ldots)
    \item Kết luận (1 - 2 câu): Nêu ý nghĩa khoa học và thực tiễn (nếu có) của phương pháp/giải pháp hay kết quả của đề tài, khả năng áp dụng vào thực tế/đóng góp cho doanh nghiệp/người dùng/nghiên cứu sau này, \ldots
\end{enumerate}